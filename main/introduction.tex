\documentclass{article}
\usepackage{mathtools}
\usepackage[margin=1in]{geometry}
\usepackage[doublespacing]{setspace}
\usepackage{enumitem}

\begin{document}
    \begin{center}
        \bf{\large Physics 182 \\ Thesis Introduction Rough Draft \\ Christopher Milke}
    \end{center}


    \vspace{10mm}
    \begin{center}
        \underline{ \large Introduction to the International Linear Collider }
    \end{center}
        \begin{center}
            \bf{What is the ILC and Why Does it Need to Exist?}
        \end{center}

        [I'm leaving this part blank for now because I want to talk about how the ILC is more precise than any other detector before it. Unfortunately, I don't have time this week to find a source that explains WHY it's more precise. Everything just says "because it uses electron/positron collisions". I'm abstaining from this section until I either find a source that actually explains why you need electron/positron collisions to be precise, or until I'm told by my advisor that I don't need to worry about it]

        \vspace{5mm}
        \begin{center}
            \bf{An Overview of the SiD}
        \end{center}

        [I'm leaving this blank because I want a bit of extra time to think about/talk to my advisor how much detail I need to go into and on what aspects of the detector]



    \vspace{10mm}
    \begin{center}
        \underline{\bf{\large Modifications to the SiD}}
    \end{center}
        \begin{center}
            \bf{L Star}
        \end{center}

        The L* of the SiD refers to the distance between the Interaction Point and the beginning of the cryostat (see figure [that doesn't exist yet]).




\end{document}
