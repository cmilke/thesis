\documentclass{article}
\usepackage{mathtools}
\usepackage[margin=1in]{geometry}
\usepackage[doublespacing]{setspace}
\usepackage{enumitem}

\begin{document}
    \begin{center}
        \bf{\large Physics 182 \\ Thesis Introduction Rough Draft \\ Christopher Milke}
    \end{center}


    \vspace{10mm}
    \begin{center}
        \underline{ \large An Introduction to the International Linear Collider }
    \end{center}
        \begin{center}
            \bf{What is the ILC and Why Does it Need to Exist?}
        \end{center}

        [I'm leaving this part blank for now because I want to talk about how the ILC is more precise than any other detector before it. Unfortunately, I don't have time this week to find a source that explains WHY it's more precise. Everything just says "because it uses electron/positron collisions". I'm abstaining from this section until I either find a source that actually explains why you need electron/positron collisions to be precise (look up the article for protons being a collection of particles), or until I'm told by my advisor that I don't need to worry about it]

        \vspace{5mm}
        \begin{center}
            \bf{An Overview of the SiD}
        \end{center}

        [I'm leaving this blank because I want a bit of extra time to think about/talk to my advisor how much detail I need to go into and on what aspects of the detector; I do know that I need to discuss the BeamCal/Vertex albedo issue, which may require that I describe how the BeamCal is a giant particle stopper]



    \vspace{10mm}
    \begin{center}
        \underline{\bf{\large Modifications to the SiD}}
    \end{center}
        \begin{center}
            \bf{L Star}
        \end{center}

        The L* of the SiD refers to the distance between the Interaction Point and the beginning of the cryostat (see figure [that doesn't exist yet]). The most recent technical design report for the SiD specified that for a number of reasons [figure out reasons; find source for this] the cryostat needed to be be moved farther away from the Interaction Point, from 3.5 meters to 4.1 meters. The relevance of this change to detector studies is that the BeamCal is attached to the cryostat, so moving the cryostat also means moving the BeamCal (from 2.95 meters to 3.265 meters). This decision has already been made, and so the studies in the analysis section are only to see what the effect of this change is; the results of this study will not affect the design decision.


        \begin{center}
            \bf{Plug Region}
        \end{center}

        As discussed previously [this will be in the 'An Overview of the SiD' section], the BeamCal is hit with a very large amount of energy. As can be seen in [find a reference picture of energy deposited on the BeamCal or make one], the overwhelming majority of the energy is deposited in the central area of the BeamCal, between the two beampipes. This central location is referred to as the plug region. Removing all or partof the plug region thus has the potential to reduce the energy incident on the face of the BeamCal, and consequently reduce the albedo effect into the Vertex Detector. There are three proposed designs for the plug region, shown in figure [insert figure of the three designs], which remove progressively more detector material. [This is where I ask for a recommendation as to how to wrap up this paragraph. Should I mention what we expect to happen? I don't like just leaving it hang as it is]


        \begin{center}
            \bf{Anti-Did Magnetic Field}
        \end{center}

        In response to the problem of albedo from the BeamCal into the Vertex Detector, an engineering team proposed the use of a magnetic field known as the DiD, as well as its counterpart, the anti-DiD [find reference to where this was proposed]. The idea of the anti-Did is to redirect low-energy background particles such that they are funneled into the outgoing beampipe of the BeamCal. By redirecting these particles, the particles hitting the BeamCal are reduced, which reduces the albedo into the Vertex Detector. However, the anti-DiD field is very expensive, and has the potential to cause a number of problems with physics analysis. Additionally, a study performed by Thomas Markiewicz [pull up the presentation that showed those results; maybe show the really important slide that was in your presentation] suggested that the anti-DiD field could potentially be increasing energy deposition outside of the plug region. For these reasons, a detailed study of the effects of the anti-DiD field on the BeamCal and Vertex Detector is necessary.



\end{document}
