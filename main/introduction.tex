\documentclass{article}
\usepackage{mathtools}
\usepackage[margin=1in]{geometry}
\usepackage[doublespacing]{setspace}
\usepackage{enumitem}

\begin{document}
    \begin{center}
        \bf{\large Physics 182 \\ Thesis Introduction Rough Draft V2 \\ Christopher Milke}
    \end{center}


    \vspace{10mm}
    \begin{center}
        \underline{ \large Introduction to the International Linear Collider }
    \end{center}
        \begin{center} \bf{What is the ILC and Why Does it Need to Exist?} \end{center}

        The International Linear Collider (ILC) is a 30 kilometer long linear particle accelerator which will collide electrons and positrons together at 500 GeV energies. It will primarily be used for studying the properties of the Higgs Boson, attempting to find new dimensions, and trying to discover Supersymetric (SUSY) particles. All three of these are already being pursued by the much more powerful Large Hadron Collider (LHC), which begs the question of why the ILC is needed. The answer is that, while the LHC is more significantly powerful, the ILC will be significantly more precise. This is because the LHC is, as the name would imply, a hadron collider. In the simplest case, the LHC performs proton-proton collisions. However, protons are not elementary particles. They consist of three quarks and any number of gluons holding those quarks together. A collision between two protons then is actually a collision between six quarks and several gluons. This is a problem for physics studies in particle accelerators, because of something known as a parton distribution function. To understand why, a brief explanation of how one conducts particle accelerator physics studies is in order.

        All the physics processes mentioned, and indeed most other physics processes studied in particle colliders, are studied by reconstructing the paths and energies of particles as they traverse the various detector elements surrounding the collider's interaction point (IP). The reconstructed paths are then compared to the initial state of the particles that went into the collision. The key statement here is that the comparison to the particles' \textit{initial} state. While the positions and energies of the protons that are colliding may be well known, the same cannot be said for the individual quarks and gluons the proton is made up of. All that can be done for the constituent particles is to make estimates on where they \textit{might} be based a probabilistic distribution, known as a Parton Distribution Function (PDF). As a result, almost all studies done at the LHC (or any other hadron collider for that matter) face a constant source of significant systematic error on all results it produces. The ILC eliminates this issue entirely by colliding only electrons and positrons, both elementary particles that have no need of PDFs. As a result, the ILC can perform physics studies at a much more precise level, providing details on physics that the LHC cannot.


        \vspace{5mm}
        \begin{center} \bf{An Overview of the SiD} \end{center}

        [I'm leaving this blank because I want a bit of extra time to think about/talk to my advisor how much detail I need to go into and on what aspects of the detector]



    \vspace{10mm}
    \begin{center}
        \underline{\bf{\large Modifications to the SiD}}
    \end{center}
        \begin{center}
            \bf{L Star}
        \end{center}

        The L* of the SiD refers to the distance between the Interaction Point and the beginning of the cryostat (see figure [that doesn't exist yet]).




\end{document}
