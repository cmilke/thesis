\documentclass{report}
\usepackage{mathtools}
\usepackage{enumitem}
\usepackage{cite}
\usepackage{titlesec}
\usepackage{graphicx}
\usepackage[margin=1in]{geometry}
\usepackage[doublespacing]{setspace}



\title{Physics BA Thesis}
\author{Christopher Don Milke}
\date{\today}


\graphicspath{ {images/} }

\titleformat{\chapter}[display] {\normalfont\bfseries}{}{0pt}{\LARGE}

\begin{document}
	\begin{titlepage}
        \maketitle
	\end{titlepage}

    \chapter{ Background }
        \section{ The International Linear Collider }
            \subsection{ What is the ILC and Why Does it Need to Exist? }
                \begin{figure}[h] 
                    \includegraphics[width=\textwidth]{ilcoverview}
                    \centering
                    \caption{Overview of the ILC}
                    \label{ilcoverview}
                \end{figure}

                The International Linear Collider (ILC) is a 30 kilometer long \cite{parton} linear particle accelerator \cite{specs} which will collide electrons and positrons together at 500 GeV energies. It will primarily be used for studying the properties of the Higgs Boson, attempting to find new dimensions, and trying to discover Supersymetric (SUSY) particles. All three of these are already being pursued by the much more powerful Large Hadron Collider (LHC), which begs the question of why the ILC is needed. The answer is that, while the LHC is significantly more powerful, the ILC will be significantly more precise. This is because the LHC is, as the name would imply, a hadron collider. In the simplest case, the LHC performs proton-proton collisions. However, protons are not elementary particles. They consist of three quarks and any number of gluons holding those quarks together. A collision between two protons then is actually a collision between six quarks and several gluons. This is a problem for physics studies in particle accelerators, because of something known as a parton distribution function. To understand why, a brief explanation of how one conducts particle accelerator physics studies is in order.

                All the physics processes mentioned, and indeed most other physics processes studied in particle colliders, are studied by reconstructing the paths and energies of particles as they traverse the various detector elements surrounding the particle collision point. The reconstructed paths are then compared to the initial state of the particles that went into the collision. The key statement here is that the comparison is to the particles' \textit{initial} state. While the positions and energies of the protons that are colliding may be well known, the same cannot be said for the individual quarks and gluons the proton is made up of. All that can be done for the constituent particles is to make estimates on where they \textit{might} be based a probabilistic distribution, known as a Parton Distribution Function (PDF). As a result, almost all studies done at the LHC (or any other hadron collider for that matter) face a constant source of significant systematic error on all results it produces. The ILC eliminates this issue entirely by colliding only electrons and positrons, both elementary particles that have no need of PDFs. As a result, the ILC can perform physics studies at a much more precise level, providing details on physics that the LHC cannot.


            \section{An Overview of the Vertex Detector and BeamCal}
                \begin{figure}[h] 
                    \includegraphics[width=\textwidth]{sid_zoom1}
                    \centering
                    \caption{Zoomed in view of the SiD.
                        The Vertex Detector is circled in red,
                        the BeamCals in magenta.}
                    \label{sid_zoom1}
                \end{figure}

                At the crossing of the positron and electron beams is the ILC's interaction point (IP). At any given time the IP can be surrounded by one of two exchangeable detectors: the Large International Detector (ILD) or the Silicon Detector (SiD). The focus in this study will be on the latter of the two. The SiD is an array of trackers and calorimeters over 12 meters long and over 16 meters in diameter. Of the numerous detectors it consists of, the two relevant to this study are the Vertex Detector and Beam Calorimeter (BeamCal). Both detectors are located extremely close to the IP of the SiD, in an area known as the Interaction Region (IR). 
                
                \begin{figure}[h]
                    \centering
                    \begin{minipage}{0.4\textwidth}
                        \includegraphics[width=\textwidth]{vertex}
                        \caption{Vertex Detector}
                        \label{vertex}
                    \end{minipage}
                    \begin{minipage}{0.4\textwidth}
                        \includegraphics[width=\textwidth]{beamcal_full}
                        \caption{Beam Calorimeter}
                        \label{beamcal}
                    \end{minipage}
                \end{figure}
                
                The Vertex Detector, with a radius of only 6.3 centimeters and a length of 5.2 centimeters, immediately surrounds the IP. The BeamCal (or rather BeamCals, as the entire SiD is mirrored) is a fair distance away from the IP, at 32.65 meters. But with a radius of 14 centimeters, it closely hugs both the incoming and outgoing beam pipes, visible as holes in figure \ref{beamcal}. The locations of the detectors provide unique advantages to each. For the Vertex Detector, surrounding the IP so closely means that it is often the first to detect particles, and is able to detect them before the solenoid's magnetic field has had much chance to deflect them. The BeamCal meanwhile sits in such a far away location, with such a small radius, that the only particles which hit it are those with extremely low transverse momenta. Generally the only particles with such a low trajectory angle are those which have had minimal interaction with other particles, and have thus retained most of their energy. As a result, the BeamCal is hit by some of the highest energy particles produced by the electron/positron collision. Besides their locations though, the Vertex Detector and BeamCal function in almost diametrically opposed ways.

                The Vertex Detector is a tracker-style detector, designed to track the positions of particles as they move through its layers. By identifying hits from a particle across the Vertex Detector's layers, the particle's trajectory can be determined. The Vertex Detector is built to interact with the particle as little as possible, so as not to interfere with the particle's trajectory. Moreover, it is critically important that the occupancy (the number of particles hitting the detector) be kept as low as possible. If the occupancy is too high, then it becomes difficult or impossible to effectively reconstruct particle trajectories. The BeamCal, on the other hand, is designed primarily for calorimetry, determining the energy a particle holds (though it does some position tracking as well). This is accomplished by interspersing silicon detector plates with large amounts of tungsten. The tungsten scatters and absorbs the energy of any particles hitting it, eventually stopping them completely. Energy is then measured by looking at how many layers of tungsten a particle was able to traverse before being stopped. This method of measuring energy is also the basis of a conflict between the Vertex Detector and BeamCal. Because the BeamCal effectively acts as a wall to all incoming particles, some of the particles actually ricochet backwards off of it, in an effect called \textit{albedo}. These showers of albedo particles rebounding off the BeamCal then proceed to fly back towards the IP, and into the Vertex Detector.
                
                The effect requires a careful balancing act. On one end of the scale, albedo increases the occupancy in the Vertex Detector, and should thus be kept to a minimum. On the other end, the BeamCal is a crucial component of the SiD, necessary for low PT (low transverse momentum) particle tagging and beam parameter checking. The BeamCal cannot be removed, but its presence negatively effects the performance of the Vertex Detector. This study aims to research the effects of three independent changes to the architecture of the Interaction Region on this balance, analyzing how each of the changes alters the effectiveness of the Vertex Detector and BeamCal. The first change is with regard to the proximity of the BeamCal to the IP, the second studies the effects of cutting out part of the BeamCal, and the third determines the usefulness of a specialized magnetic field known as the anti-DiD.


        \section{Modifications to the SiD}
            \subsection{center}

                The L* of the SiD is a term which refers to the distance between the Interaction Point and the beginning of the cryostat. The most recent technical design report for the SiD specified that for a number of reasons the cryostat needed to be be moved farther away from the Interaction Point, from 3.5 meters to 4.1 meters. The relevance of this change to detector studies is that the BeamCal is attached to the cryostat, so moving the cryostat also means moving the BeamCal (from 2.95 meters to 3.265 meters). This decision has already been made, and so the studies in the analysis section are only to see what the effect of this change is; the results of this study will not affect the design decision.


            \subsection{Plug Region}
                \begin{figure}[h]
                    \centering
                    \begin{minipage}{0.3\textwidth}
                        \includegraphics[width=\textwidth]{beamcal_plug}
                        \caption{Full plug region}
                        \label{beamcal_plug}
                    \end{minipage}
                    \begin{minipage}{0.3\textwidth}
                        \includegraphics[width=\textwidth]{beamcal_wedge}
                        \caption{Wedge cutout}
                        \label{beamcal_wedge}
                    \end{minipage}
                    \begin{minipage}{0.3\textwidth}
                        \includegraphics[width=\textwidth]{beamcal_circle}
                        \caption{Circle cutout}
                        \label{beamcal_circle}
                    \end{minipage}
                \end{figure}

                As discussed previously, the BeamCal is hit with a very large amount of energy. The overwhelming majority of the energy is deposited in the central area of the BeamCal, between the two beam pipes, as apparent in figure [make a figure for this plz]. This central location is referred to as the plug region. Removing all or part of the plug region thus has the potential to reduce the energy incident on the face of the BeamCal, and consequently reduce the albedo effect into the Vertex Detector. There are three proposed designs for the plug region (shown in figures \ref{beamcal_plug}, \ref{beamcal_wedge}, and \ref{beamcal_circle}) which remove progressively more detector material. In theory, these designs should provide lower occupancy in the vertex detector, but the cost to the BeamCal is not yet known.


            \subsection{Anti-Did Magnetic Field}

                In response to the problem of albedo from the BeamCal into the Vertex Detector, an engineering team proposed the use of a magnetic field known as the DiD, as well as its counterpart, the anti-DiD. The idea of the anti-Did is to redirect low-energy background particles such that they are funneled into the outgoing beampipe of the BeamCal. By redirecting these particles, the particles hitting the BeamCal are reduced, which reduces the albedo into the Vertex Detector. However, the anti-DiD field is very expensive, and has the potential to cause a number of problems with physics analysis. Additionally, a study performed by Thomas Markiewicz \cite{anti-did} suggested that the anti-DiD field could actually be increasing energy deposition outside of the plug region, only causing improvement in a limited area of the detector. For these reasons, a detailed study of the effects of the anti-DiD field on the BeamCal and Vertex Detector is necessary.


        \section{Goals of the Study}

            The objective of this study will be to update the Interaction Region of the SiD to the most modern and up-to-date specifications, directly using the designs of the SiD's top engineer \cite{excel}. Following from this, the three aforementioned changes will be implemented one at a time, analyzing the effects on the performance of the Vertex Detector and Beam Calorimeter. 





    \chapter{Creating New SiD Models}
        \section{Updating the Old SiD Model}
            
            The ultimate goal of this study was to analyze new modification to the SiD model. However, before we could look into the future of the SiD, we had to take it out of the past. Prior to my modifications, the most up-to-date simulation model of the SiD was the SiDloi3 model. However, the SiDloi3 model was written in 2011, and numerous changes to the SiD's engineering design had been made since, rendering SiDloi3 outdated. Thus, the first order of business was to ensure that the simulation model was matched to the current engineering design; at least, the aspects of it we were concerned with. As discussed in the introduction, this study focused exclusively on the interaction region of the detector, and as such only the interaction region was fully updated. But even with the scope of the project restricted, the update was no simple task.

            The most pressing change by far was the realignment of the BeamCal. In the SiDloi3 model, the BeamCal was designed to be concentric about the simulation space's z-axis (figure [beamcalZaligned]). The most recent engineering designs however (provided by Thomas Markievicz), place the BeamCal concentric about the outgoing beampipe (figure [beamcalOutaligned]). In general, modifications to the SiD model are made by editing a specific xml file titled "compact.xml". The modified compact is then run through a program called GeomConverter, which is an extension of the Java based LCSim (Linear Collider Simulator) framework. However, the GeomConverter package did not support off-center alignments of any detector elements, requiring significant modifications be made to the source code (the full source changelog can be found in appendix 1A). Once the changes were made, and the BeamCal appropriately realigned to be concentric about the outgoing beampipe, it was a simple matter to similarly realign the BeamCal's associated masks: the ForwardLowZ mask (figure [forwardlowz]), and the ForwardM1 mask (figure [forwardm1]). 

            Hypothetically speaking, once the BeamCal and its masks were properly realigned, the study could have proceeded forward without any further updates to the SiD model. Unfortunately, usage of CERN ROOT's gdml viewer revealed that the realignment of the BeamCal caused it to collide with a support tube structure (figure [supportTube]). It was quickly apparent that altering the support tube structure alone would be useless, as this would result in even more collisions (figure [supportTube2]). Thus, additional modifications to the detector elements surounding the support (notably the Ecal, Hcal, and Muon Detector) were needed. While these modifications were in line with Markievicz's engineering designs, they only constitute partial modifications to the non-interaction region elements.

            In addition to the BeamCal, it was also decided that the LumiCal (figure 1.6) should also be updated. The reason for this is that the LumiCal places constraints on the angles at which particles can hit the BeamCal (figure 1.7). Updating the LumiCal brought about a problem of its own though. In the most recent engineering designs, the LumiCal was moved closer to the interaction point, such that the outermost layer of the SiTracker Endcap surrounded it (figure 1.8). While this poses no realistic problem, the GEANT4 simulation software would not tolerate it, as this placed the LumiCal within a virtual boundry called the "Tracking Volume". In short, the Tracking Volume causes all elements inside it to store significantly more simulation information. Because the LumiCal is a calorimeter, and therefore induces a large shower of particles, storing the additional information from the Tracking Volume caused excessive RAM usage, making simulations impossible. As a result, we were forced to move the outermost layer of the SiTracker endcap closer to the interaction point, so that it no longer surrounded the LumiCal. Upon finishing this adjustment though, the SiD forward region was fully up to date, and new design decisions could be implemented (the full geometry changelog can be found in appendix 1B).


        \section{Experimenting with New SiD Models}

            With the sidloi3 sufficiently updated, experiments on the various alternate configurations could begin. The three geometric changes, as discussed in the introduction, are the length of the L*, the configuration of the plug region of the BeamCal, and the inclusion of an anti-did field. Six lcdd simulation geometries were generated (using the same tools discussed in the previous section) in order to analyze these changes, with figure [geomChart] displaying the range of configurations. For the two detector elements studied (BeamCal and Vertex Detector), two simulation methods were used. 
            
            For the BeamCal, these lcdd files were run through SLIC (a GEANT4 wrapper), on SLAC's LSF computing cluster, in order to generate 300 pairbackground events and 10,000 ad-hoc high energy signal events per geometric configuration. The discrepency in event numbers is due to the high time demands of generating pairbackgrounds, coupled with the fact that pairbackgrounds are not the main focus of the BeamCal study (thus requiring less rigorous statistics). It is worth mentioning here just what these 'ad-hoc' electrons are. They are high energy electrons that are generated at the IP with extremely low transverse momentum. They are \textit{not} real, physical events. They are merely a useful tool for first order studies of a detector's tagging efficiency.

            The simulations performed for the Vertex detector underwent a different process. The studies for the Vertex Detector are solely dependant on pairbackground events, and demand the rigorous statistics the BeamCal was able to avoid. As already mentioned, pairbackgrounds are very computationaly intensive and time consuming to generate, to a degree the SLAC cluster is unequipped for. As such, I turned to PNNL physicist and SiD tech supportor [figure out the right title] Jan Strube. Jan then used the Open Science Grid (OSG) to generate 2500 pairbackground events per geometric configuration (2500 being the number of bunch crossings in a luminosity upgrade train). Additionally, in order to save hard disk space, Jan stripped the pairbackground simulation files of all information except that pertaining to the Vertex Detector. 

            The end result of both simulation methods was a large collection of slcio files, which contained the details of the various simulations. From here, the LCSim framework was used as a backend to a personal suite of analysis code I wrote to study the physics of the BeamCal and Vertex Detector. For the Vertex Detector, a program was used to pixellate the hits on the detector and then perform fractional occupancy analysis on the resulting pixels as function of various parameters (discussed further in the Analysis section). For the BeamCal, the analysis is more complicated.
            
            The analysis of the BeamCal studies the efficiency of its signal reconstruction efficiency. It is appropriate then to provide an overview of how the reconstruction algorithm works (a more thorough discussion can be found in reference [Bogert thesis]). The algorithm first populates the pixels of the BeamCalorimeter with the information provided by both the signal electron and the pairbackground lcdd files, one event at a time. It then performs a sweep across the tenth layer of the BeamCal, aggregating the pixels into clusters called 'pallets'. It selects the pallets with the top 50 highest background-subtracted energies deposited in them. The selected pallets are then 'extended' to include all pixels in the same x/y locations in layers 10 through 40, summing the raw energy of all included pixels to produce a new structure refered to as a 'cylinder'. The 50 cylinders are iterated through. If the background-subtracted energy of any of the cylinders exceeds a particular sigma cut, then the event is deemed to contain a signal event and rejected. The sigma cut is chosen such that no more than ten percent of all events which do \textit{not} contain a signal event are rejected. This algorithm's efficiency in correctly identifying signal electrons, particularly as a function of radius, was then studied and compared across the range of architectural configurations, using three different measures of efficiency.

            By measures of efficiency, I mean \textit{instrumental} efficiency and \textit{geometric} efficiency. Instrumental efficiency is the frequency with which the tagging algorithm correctly identifies events where the high energy electron/positron \textit{hits the BeamCal}. If the electron/positron misses the BeamCal entirely then the algorithm is unable to tag it. In instrumental efficiency though, only events where the electron/positron hit the BeamCal count towards the efficiency rating. Conversly, geometric efficiency ignores the tagging algorithm entirely, determining only whether or not an electron/positron hit the BeamCal. The combination of these two, the frequency of how often the algorithm correctly tags \textit{any} electron/positron (including those which miss the BeamCal), is reffered to as the total efficiency. In the following analysis, the total efficiency of the architectures is studied, and then disected by separetly comparing the instrumental and geometric efficiencies.






    \chapter{Analysis}
        \section{L star}
            \subsection{Beam Calorimeter}
                As visible in figure [l* beamcal total compare plot], moving the BeamCal further away from the IP only has a marginal effect, except at the innermost radii. Looking at figures [l* geom compare plot] and [l* inst compare], it is apparent that this is a geometric effect.
                
            \subsection{Vertex Detector} 
                The effect of the BeamCal movement on the Vertex Detector Occupancy is effectively non-existent. Though the barrel's occupancy dropped slightly, the endcap's increased by roughly the same amount. A possible explanation for this is that the number of particles coming towards the Vertex Detector is the same, with the only change being the radius at which the particles impacted. That is, with the BeamCal moved further back, the background particles must travel further to reach it and then return, which results in a larger transverse displacement. Thus, the number of background particles is left unchanged but the particles now hit further out in radius (shown in figure [make a figure]). The Vertex Detector Endcaps extend further out in radius than the barrels, hence why the endcaps occupancy grew while the barrels dropped.

            \subsection{Conclusion}


        \section{Plug Region}
            \subsection{Beam Calorimeter}
                As expected, removing part of the BeamCal causes massive drops in performance near the removed region, as visible in figure [you know the drill]. Additionally, the performance drop increases with the more liberal circular cutout. Outside of the inner radii, the performance is unchanged, as was also expected. A minor point worth explaining is why performance drops slightly beyond the excised region. [do some math, look at the molier radius. That's probably why].
                
            \subsection{Vertex Detector} 
                The backscatter effect from the BeamCal into the Vertex Detector is dependent on there being something for the particles to scatter backwards off of. It makes perfect sense then that removing material from the BeamCal would decrease the occupancy in the Vertex Detector, visible in figure [yadda yadda], with the circle cutout resulting in even lower occupancy levels.
                

            \subsection{Conclusion} 
                 It would certainly prove beneficial to the Vertex Detector's performance if part of the plug region can be removed, but whether or not the losses to the BeamCal are tolerable is still an open question. The events used to analyze them in this study are, as mentioned previously, not real physical events. The event type that this algorithm really needs to be used for is the gamma-gamma to hadron event type. At the time of writing this, studies on the gamma-gamma to hadron events have begun, but are many months from completion. A more thorough decision on the plug region will be made once the plug region's effect on gamma-gamma to hadron tagging is studied.

        
        \section{Anti-DiD}
            \subsection{Beam Calorimeter}

            \subsection{Vertex Detector} 

            \subsection{Conclusion}







    \bibliography{ref.bib}    
    \bibliographystyle{apalike}


\end{document}
