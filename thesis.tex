\documentclass{report}
\usepackage{textcomp}
\usepackage{mathtools}
\usepackage{enumitem}
\usepackage{cite}
\usepackage{titlesec}
\usepackage{graphicx}
\usepackage{float}
\usepackage[toc,page]{appendix}
\usepackage[margin=1in]{geometry}
\usepackage[doublespacing]{setspace}
\usepackage[font=scriptsize]{caption}



\graphicspath{ {images/} }
\widowpenalty=1500
\clubpenalty=1500

\titlespacing*{\chapter}{0pt}{0pt}{20pt}
\titleformat{\chapter}[display] {\normalfont\bfseries}{}{0pt}{\centering \LARGE}

\begin{document}
	\begin{titlepage} \begin{singlespace}
        \begin{center} \begin{noindent} \begin{large}
            UNIVERSITY of CALIFORNIA \\ SANTA CRUZ

            \vspace{\baselineskip}

            \textbf{INTERNATIONAL LINEAR COLLIDER: \\
                PERFORMANCE OF THE BEAM CALORIMETER \\ 
                AND SI-VERTEX DETECTOR \\
                IN LIGHT OF ARCHITECTURAL CHANGES TO THE SID}

            \vspace{\baselineskip}
            A thesis submitted in partial satisfaction of the \\ requirements for the degree of \\
            \vspace{\baselineskip}
            BACHELOR OF SCIENCE \\
            \vspace{\baselineskip}
            in \\
            \vspace{\baselineskip}
            PHYSICS \\
            \vspace{\baselineskip}
            by \\
            \vspace{\baselineskip}
            \textbf{Christopher D. Milke} \\
            \vspace{\baselineskip}

            \today \\
            \vspace*{\fill}

            The thesis of Christopher Milke is approved by: \\
            \end{large}
            \vspace{1.5cm}
            \rule{70mm}{.5pt} \hfill \rule{70mm}{.5pt} \\
            Professor Bruce Schumm \hfill Adriane Steinacker \\
            Adviser \hfill Thesis Coordinator \\
            \vspace{1.5cm}
            \rule{70mm}{.5pt} \\
            Robert Johnson \\
            Chair, Department of Physics \\
        \end{noindent} \end{center}

	\end{singlespace} \end{titlepage}


    %copyright
    \newpage \begin{center} \pagenumbering{roman}
        \vspace*{\fill}
        Copyright \textcopyright by

        Christopher D. Milke 

        2016
        \vspace*{\fill}
    \end{center} \newpage


    %abstract
    \addcontentsline{toc}{chapter}{Abstract}
        \begin{center} \LARGE \textbf{Abstract} \end{center}

        In ILC development, there are a number of possible alternatives to the forward region of the SiD. Using a combination of Geant4, ROOT, and LCSIM, these alterations will be studied in their effect on the Beam Calorimeter subsystem's electron tagging efficiency, and on the occupancy of the SiVertex detector subsystem (both Barrel and Endcap).

        The first alteration considered is the adjustment of the length of the L* (the distance from the final focusing magnet to the interaction point) from 3.5 meters to 4.2 meters. In regards to my study, this adjustment translates to shifting the BeamCal 70 cm further away from the interaction point. This change was found to increase the BeamCal's efficiency by roughly 5\% in the inner radii, with no noticeable change at outer radii. In the Vertex Detector, it was found that the albedo picked up by the Vertex Detector remained unchanged overall. The movement of the BeamCal merely altered the deflection angle off of its face, slightly altering (but not reducing) the occupancy distribution within the Vertex Detector.

        The second alternative is how the $e^+/e^-$ particle beam pipes pass through the BeamCal. Three methods are considered: a separate hole in the BeamCal for each pipe, a single wedge shaped hole which removes the material between the beam pipes, and a single larger hole which circumscribes the two beam pipes. BeamCal efficiency at low radii were severely impacted as more material was removed. Conversely, Vertex detector occupancy improved with the more drastic cuts, but only by marginal amounts.

        The final alteration is in whether or not there should be an anti-DiD field superimposed upon the SiD solenoidal field to guide beam induced background into the beampipe holes. The addition of the anti-DiD field improved both BeamCal efficiency and Vertex Detector occupancy. However, the improvements were small, and not considered worth the investment of installing an anti-DiD field.
    \newpage


    \tableofcontents


    %dedication and acknowledgments
    \newpage \vspace*{\fill}
        \addcontentsline{toc}{chapter}{Dedication}
        \begin{center} \begin{large}
            To my parents, 
            
            without whose unconditional love and support
            
            I would never be where I am today.

            \vspace{40mm}

            And to the best team of undergraduate physicists
            
            I could ever have asked for.

            The ILC project is in your hands now.
            
            I know you will all see it through to success.
        \end{large} \end{center}
    \vspace*{\fill} \newpage \vspace*{\fill}
        \addcontentsline{toc}{chapter}{Acknowledgments}
        \begin{center} \begin{large}
            \large \textbf{Acknowledgments} \vspace{\baselineskip}

            I would like to first and foremost thank my adviser, professor Bruce Schumm, who was a constant source of guidance through this project in particular, and my career as a physicist in general. I also want to thank Jan Strube of PNNL for helping me find my way through the labyrinth of code frameworks and programs and servers that constituted this study. Additionally, I wish to thank the entirety of the SiD Collaboration, which provided a wealth of information and set this project into motion. I want to thank Jason Nielsen for providing us access to his ATLAS server. And finally, I want to thank Luc D'Hauthuille, who spent many long hours working alongside me to crank out the results of this study.
        \end{large} \end{center}
    \vspace*{\fill} \newpage





    \chapter{ Background }
        \pagenumbering{arabic} \setcounter{page}{1}
        \section{ The International Linear Collider }
            %in: None
            %out: Size of ILC, e+/- collider, collision energy, uses, ILC is precise b/c of parton distributions 
            \subsection{ What is the ILC and Why Do We Want it to be Built? }
                \begin{figure}[h] 
                    \includegraphics[width=\textwidth]{ilcoverview}
                    \centering
                    \caption{Overview of the ILC}
                    \label{fig__ilcoverview}
                \end{figure}

                The International Linear Collider (ILC) is a 30 kilometer long linear particle accelerator \cite{specs} which will collide electrons and positrons together at 500 GeV energies. It will primarily be used for studying the properties of the Higgs Boson, attempting to find new dimensions, and trying to discover Supersymetric (SUSY) particles. All three of these are already being pursued by the much more powerful Large Hadron Collider (LHC), which begs the question of why the ILC is needed. The answer is that, while the LHC is significantly more powerful, the ILC will be significantly more precise. This is because the LHC is, as the name would imply, a hadron collider. In the simplest case, the LHC performs proton-proton collisions. However, protons are not elementary particles. They consist of three quarks and any number of gluons holding those quarks together. A collision between two protons then is actually a collision between six quarks and several gluons. This is a problem for physics studies in particle accelerators, because of something known as a parton distribution function \cite{parton}. To understand why, a brief explanation of how one conducts particle accelerator physics studies is in order.

                All the physics processes mentioned, and indeed most other physics processes studied in particle colliders, are studied by reconstructing the paths and energies of particles as they traverse the various detector elements surrounding the particle collision point. The reconstructed paths are then often compared to the initial state of the particles that went into the collision. The key statement here is that the comparison is to the particles' \textit{initial} state. While the positions and energies of the protons that are colliding may be well known, the same cannot be said for the individual quarks and gluons the proton is made up of. All that can be done for the constituent particles is to make estimates on where they \textit{might} be based on a probabilistic distribution, known as a Parton Distribution Function (PDF). As a result, many precision studies done at the LHC face a constant source of significant uncertainty on a number of results they might produce. The ILC eliminates this issue entirely, by colliding only electrons and positrons. These are both elementary particles, and hence their initial states are precisely known, eliminating the need for PDFs. 

                To be more specific, the ILC collides a 10.8 Mega-watt electron beam with a 10.8 Mega-watt positron beam, with a center-of-mass energy of 500 GeV and a peak luminosity of $ 2 * 10^{34} cm^{-2}s^{-1} $. The collision of the beams is not continuous, but rather comes in 5 Hz pulses. Each pulse delivers a train of bunch crossings. A bunch crossing is 3.2 nano-Coulombs worth of electrons and positrons colliding at once, and a nominal train is 2,625 bunch crossings. At the point of collision, the electron and positron beams will be focused to a size of 640 nm by 5.7 nm, crossing at an angle of 14 mili-radians \cite{beam_specs}. All told, the result of these parameters will result in the ILC's ability to perform physics studies at a much more precise level, providing details on physics that the LHC cannot.

            %in: ILC context
            %out: tons of info on vertex and beamcal, albedo conflict, geom change list
            \subsection{An Overview of the Vertex Detector and BeamCal}
                \begin{figure}[h] 
                    \includegraphics[width=\textwidth]{sid_zoom1}
                    \centering
                    \caption{Zoomed in view of the SiD.
                        The Vertex Detector is circled in red,
                        the BeamCals in magenta.}
                    \label{fig__sid_zoom1}
                \end{figure}

                %in: What the ILC is
                %out: What is IP, what is SiD, size of SiD, location of Vertex and BeamCal,
                At the crossing of the positron and electron beams is the ILC's interaction point (IP). At any given time the IP can be surrounded by one of two exchangeable detectors: the Large International Detector (ILD) or the Silicon Detector (SiD). The focus in this study will be on the latter of the two. The SiD is an array of trackers and calorimeters over 12 meters long and over 16 meters in diameter. Of the numerous detectors it consists of, the two relevant to this study are the Vertex Detector and Beam Calorimeter (BeamCal). Both detectors are located extremely close to the IP of the SiD, in an area known as the Interaction Region (IR). 

                \begin{figure}[h]
                    \centering
                    \begin{minipage}{0.4\textwidth}
                        \includegraphics[width=\textwidth]{vertex}
                        \caption{Vertex Detector - The flat disks at either end are the Endcaps,
                                 and the concentric cylinders in the center comprise the Barrel.}
                        \label{fig__vertex}
                    \end{minipage}
                    \begin{minipage}{0.4\textwidth}
                        \includegraphics[width=\textwidth]{beamcal_full}
                        \caption{Beam Calorimeter - Note that in this figure the BeamCal is centered around the outgoing beam pipe. }
                        \label{fig__beamcal}
                    \end{minipage}
                \end{figure}
                
                %in: context of vertex & beamcal 
                %out: location of Vertex and BeamCal, size of vertex, size of BeamCal, use of Vertex, advantages of vertex, beamcal only hit with lowpt particles, lowpt particles are often high energy, beamcal gets a lot of bgd energy, use of Vertex, advantages of vertex, beamcal only hit with lowpt particles, lowpt particles are often high energy, beamcal gets a lot of bgd energy
                The Vertex Detector, with a length of 36 centimeters and an inner radius of only 16 milimeters, immediately surrounds the IP. The BeamCal (or rather pair of BeamCals, as the entire SiD is mirrored across the IP) is a fair distance away from the IP, at 3.265 meters. But with a radius of 14 centimeters, it closely hugs both the incoming and outgoing beam pipes, visible as holes in figure \ref{fig__beamcal}. The locations of the detectors provide unique advantages to each. For the Vertex Detector, surrounding the IP so closely means that it is able to provide the most precise constraints on the path of charged particles, using some of the most advanced position-sensing detectors to perform this task. The BeamCal meanwhile sits in such a far away location, with such a small radius, that the only particles to hit it are those with an extremely small angle relative to the beamline. Often, the only beam particles with such a low trajectory angle are those which have had minimal interaction with other particles, and have thus retained most of their energy. As a result, the BeamCal is hit by some of the highest energy particles produced by the electron/positron collision. Aside from their near-beam pipe locations though, the Vertex Detector and BeamCal function in almost diametrically opposed ways.

                %in: what vertex & beamcal are
                %out: how vertex works, importance of low vertex occupancy, how beamcal works, intro to albedo problem
                The Vertex Detector is a tracker-style detector, designed to track the positions of individual particles as they move through its layers. By identifying hits from a particle across the Vertex Detector's layers, the particle's trajectory can be determined. Broken into two distinct parts, the Barrel and the Endcap, it is able to attain complete coverage of the IP at all angles. The Vertex Detector is built to interact with the particle as little as possible, so as not to deflect the particle's trajectory. Moreover, the Vertex Detector uses very small pixels, on the order of tens of microns, in order achieve the highest possible position resolution. In order to maintain this resolution though, it is critically important that the occupancy (the fraction of channels hit over a specific collection time) be kept as low as possible. As the occupancy increases, so too does the density of pixels that were hit. And as the hit pixel density increases, it becomes increasingly difficult to identify individual particles, making it much harder to effectively reconstruct particle trajectories.
                
                The BeamCal, on the other hand, is designed in part to perform some crude position tracking, but is primarily meant for calorimetry, the process of determining the energy of a particle by completely absorbing it. This is accomplished by interspersing silicon detector plates with large amounts of tungsten. The tungsten scatters and absorbs the energy of any particles hitting it, eventually stopping them completely. Energy is then measured by looking at how many particles pass through the silicon layers between the tungsten plates as the scattering shower develops. This method of measuring energy is also the basis of a conflict between the Vertex Detector and BeamCal. Because the BeamCal effectively acts as a wall to all incoming particles, some of the particles produced when the beams collide actually emanate backwards from it, an effect called \textit{albedo}. These showers of albedo particles exuding off the BeamCal then proceed to fly back towards the IP, with many of them reaching the layers of the Vertex Detector.
                
                %in: what is albedo, vertex, and beamcal
                %out: conflict between beamcal and vertex, list three changes
                The consequences of the BeamCal albedo requires a careful study of the balance between the Vertex and BeamCal performance. On one end of the scale, albedo increases the occupancy in the Vertex Detector, and should thus be kept to a minimum. On the other end, the BeamCal is a crucial component of the SiD, necessary for low transverse momentum particle tagging and beam parameter checking. The BeamCal cannot be removed, but its presence can negatively effect the performance of the Vertex Detector. This study aims to research the effects of three independent changes to the architecture of the Interaction Region on this balance, analyzing how each of the changes alters the effectiveness of the Vertex Detector and BeamCal. The first change is with regard to the proximity of the BeamCal to the IP, the second studies the effects of cutting out part of the BeamCal, and the third determines the usefulness of a specialized magnetic field known as the anti-DiD.



        \section{Modifications to the SiD} \label{sect__sid_mods}
            %in: what is sid, IP, beamcal
            %out: what is L*, relevance to beamcal
            \subsection{L Star}

                The L* of the SiD is a term which refers to the distance between the Interaction Point and the beginning of the final focussing magnet of the beamline (known as the QD0). The most recent technical design report for the SiD specified that, for a number of reasons, the magnet needed to be be moved farther away from the Interaction Point, from 3.5 meters to 4.1 meters. The relevance of this change to detector studies is that the BeamCal is attached to the QD0 magnet, so moving the magnet also means moving the BeamCal (from 2.95 meters to 3.265 meters). It is important to study the consequences of this change, as a catastrophic performance degradation could require the decision be reevaluated.
                

            %in: context of beamcal, albedo
            %out: energy distribution on beamcal, three plug designs, why designs might help and hurt
            \subsection{Plug Region}
                \begin{figure}[h]
                    \centering
                    \begin{minipage}{0.3\textwidth}
                        \includegraphics[width=\textwidth]{beamcal_plug}
                    \end{minipage}
                    \begin{minipage}{0.3\textwidth}
                        \includegraphics[width=\textwidth]{beamcal_wedge}
                    \end{minipage}
                    \begin{minipage}{0.3\textwidth}
                        \includegraphics[width=\textwidth]{beamcal_circle}
                    \end{minipage}
                    \caption{Front face of the BeamCal, showing the three different plug region implementations.}
                    \label{fig__beamcal_face}

                \end{figure}

                \begin{figure}[h]
                    \centering
                    \includegraphics[width=\textwidth]{beamcal_energy_xy}
                    \caption{Energy Deposited on BeamCal from Pairbackground event}
                    \label{fig__beamcal_energy_xy}
                \end{figure}

                As discussed previously, the BeamCal is hit with a very large amount of energy. The overwhelming majority of the energy is deposited in the central area of the BeamCal, between the two beam pipes, as apparent in figure \ref{fig__beamcal_energy_xy}. This central location is referred to as the plug region. Removing all or part of the plug region thus has the potential to reduce the energy incident on the face of the BeamCal, and consequently reduce the albedo effect into the Vertex Detector. There are three proposed designs for the plug region (shown in figure \ref{fig__beamcal_face}) which remove progressively more detector material. In theory, the designs with less material should provide lower occupancy in the vertex detector. However, it will also reduce the area over which the scattered beam particles can be detected, reducing the BeamCal's performance.


            %in: albedo, vertex, beamcal
            %out: what is anti-did, why we want it/don't want it, Tom's study
            \subsection{Anti-Did Magnetic Field}
                In response to the problem of albedo from the BeamCal into the Vertex Detector, an engineering team proposed the use of a magnetic field known as the anti-DiD. The idea of the anti-DiD is to redirect low-energy background particles such that they are funneled into the outgoing beam pipe of the BeamCal. By redirecting these particles, the particles hitting the BeamCal are reduced, which reduces the albedo into the Vertex Detector. However, the anti-DiD field is very expensive, with estimated costs in the millions of dollars. Additionally, a study performed by Thomas Markiewicz \cite{anti-did} suggested that the anti-DiD field could actually be increasing energy deposition outside of the plug region, only causing improvement in a limited area of the detector. For these reasons, a detailed study of the effects of the anti-DiD field on the BeamCal and Vertex Detector is necessary.





    \chapter{Creating New SiD Models}
        \section{Updating the Old SiD Model}
            
            %in: what is sid, what are we doing (geom changes, vertex/beamcal study)
            %out: need to update sidloi3
            The ultimate goal of this study was to analyze new modification to the SiD model. However, before we could look into the future of the SiD, we had to take it out of the past. Prior to my modifications, the most up-to-date simulation model of the SiD was the SiDloi3 model. However, the SiDloi3 model was specified in 2011, and numerous changes to the SiD's engineering design had been made since, rendering SiDloi3 outdated. Thus, the first order of business was to ensure that the simulation model was matched to the current engineering design; at least, the aspects of it we were concerned with. As discussed in the introduction, this study focused exclusively on the interaction region of the detector, and as such only the interaction region was fully updated. But even with the scope of the project restricted, the update was no simple task.

            \begin{figure}[H] 
                \includegraphics[width=0.7\textwidth]{beamcal_zaxis}
                \centering
                \caption{BeamCalorimeter concentric about z-axis}
                \label{fig__beamcal_zaxis}
            \end{figure}
            \begin{figure}[H] 
                \includegraphics[width=0.7\textwidth]{beamcal_aligned}
                \centering
                \caption{BeamCalorimeter concentric about outgoing beam pipe.}
                \label{fig__beamcal_aligned}
            \end{figure}
            \begin{figure}[H] 
                \includegraphics[width=0.7\textwidth]{beamcal_masks}
                \centering
                \caption{BeamCalorimeter with associated masks}
                \label{fig__beamcal_masks}
            \end{figure}

            %in: beamcal
            %out: beamcal alignment, geometry creation process, necessity of code altering
            The most pressing change by far was the realignment of the BeamCal. In the SiDloi3 model, the BeamCal was designed to be concentric about the simulation space's z-axis (figure \ref{fig__beamcal_zaxis}). The most recent engineering designs however (provided by Thomas Markiewicz), place the BeamCal concentric about the outgoing beam pipe (figure \ref{fig__beamcal_aligned}). In general, modifications to the SiD model are made by editing a specific xml file titled "compact.xml". The modified compact is then run through a program called GeomConverter, which is an extension of the Java based LCSim (Linear Collider Simulator) framework. However, the GeomConverter package did not support off-center alignments of any detector elements, requiring significant modifications be made to the source code (the full source code changelog can be found in section \ref{sect__geom_changes}). Once the changes were made, and the BeamCal appropriately realigned to be concentric about the outgoing beam pipe, it was a simple matter to similarly realign the BeamCal's associated masks: the ForwardLowZ mask, and the ForwardM1 mask, both visible in figure \ref{fig__beamcal_masks}). 

            \begin{figure}[h] 
                \includegraphics[width=0.7\textwidth]{beamcal_collide}
                \centering
                \caption{BeamCalorimeter Colliding with Forward Support Tube}
                \label{fig__beamcal_collide}
            \end{figure}

            %in: beamcal, need for code altering, Tom's engineering designs, brief knowledge of ecal, hcal, and muon detector
            %out: I had to change extra stuff in xml
            Hypothetically speaking, once the BeamCal and its masks were properly realigned, the study could have proceeded forward without any further updates to the SiD model. Unfortunately, usage of CERN ROOT's gdml viewer revealed that the realignment of the BeamCal caused it to collide with a support tube structure (figure \ref{fig__beamcal_collide}). It was quickly apparent that altering the support tube structure alone would be useless, as this would result in even more collisions with a myriad of other SiD components. Thus, additional modifications to the detector elements surrounding the support tube (notably the Ecal, Hcal, and Muon Detector) were needed. While these modifications were in line with Markiewicz's engineering designs, they only constitute partial modifications to the non-interaction region elements. Most of the non-interaction region elements are still using outdated designs.

            %in: beamcal, need for code altering, lumical, sitracker.
            %out: I had to change even more extra stuff in detector
            In addition to the BeamCal, it was also decided that the LumiCal should be updated. The reason for this is that the LumiCal places constraints on the angles at which particles can hit the BeamCal. Updating the LumiCal brought about a problem of its own though. In the most recent engineering designs, the LumiCal was moved closer to the interaction point, such that the outermost layer of the SiTracker Endcap surrounded it. While this poses no realistic problem, the GEANT4 simulation software would not tolerate it, as this placed the LumiCal within a virtual boundary called the "Tracking Volume", which surrounds the SiTracker. In short, the Tracking Volume causes all elements inside it to store significantly more simulation information. Because the LumiCal is a calorimeter, and therefore induces a large shower of particles, storing the additional information from the Tracking Volume caused excessive RAM usage, making simulations impossible. As a result, we were forced to move the outermost layer of the SiTracker Endcap closer to the interaction point. This allowed us to shrink the Tracking Volume so that it no longer surrounded the LumiCal. Upon finishing this adjustment, the SiD forward region was fully up to date, and new design decisions could be implemented (the full geometry changelog can be found in section \ref{sect__compact_changes}).


        \section{Experimenting with the New SiDloi3-IR\_realign Models}
            With the sidloi3 sufficiently updated, experiments on the various alternate configurations could begin. The three geometric changes, as discussed in section \ref{sect__sid_mods}, are the length of the L*, the configuration of the plug region of the BeamCal, and the inclusion of an anti-DiD field. Five lcdd simulation geometries were generated (using the same tools discussed in the previous section) in order to analyze these changes. The 'control' case was using the new 4.1 meter L* with the full plug region and no anti-DiD field. The other four were variations on this. One used the 3.5 meter L*, one used the anti-DiD field, and the remaining two used the alternate plug regions. For the two detector elements studied (BeamCal and Vertex Detector), two separate simulation methods were used. 

            For the BeamCal, these lcdd files were run through SLIC (a GEANT4 wrapper), on SLAC's LSF computing cluster, in order to generate 300 pair background events and 10,000 ad-hoc high energy signal events per geometric configuration. The discrepancy in event numbers is due to the high time demands of generating pair backgrounds, coupled with the fact that pair backgrounds are not the main focus of the BeamCal study (thus requiring less rigorous statistics). It is worth mentioning here just what these "ad-hoc" electrons are. They are high energy electrons that are generated at the IP with a distribution in the cosine of the azimuthal angle ($\cos\theta$) that is restricted to roughly correspond to the angular coverage of the BeamCal.
            
            They are \textit{not} real, physical events. They are merely a useful tool for first order studies of a detector's tagging efficiency.

            The simulations performed for the Vertex detector underwent a different process. The studies for the Vertex Detector are solely dependent on pair background events, and benefit greatly from the full statistics the BeamCal study avoided. As already mentioned, pair backgrounds are very computationally intensive and time consuming to generate, to a degree the SLAC cluster is unequipped for. As such, I turned to PNNL physicist and SiD tech support Jan Strube. Jan then used the Open Science Grid to generate 2500 pair background events per geometric configuration (2500 being the rough number of bunch crossings in a train, post luminosity upgrade). Additionally, in order to save hard disk space, Jan stripped the pair background simulation files of all information except that pertaining to the Vertex Detector. 





    \chapter{Analyzing the Simulations}
        The end result of both simulation methods was a large collection of slcio files, which contained the details of the various simulations. From here, the LCSim framework was used as a backend to a personal suite of analysis code I wrote to study the physics of the BeamCal and Vertex Detector.


        \section{Beam Calorimeter}
            The analysis of the BeamCal studies the efficiency of its signal reconstruction efficiency. By \textit{signal}, I mean high energy electrons or positrons passing through the BeamCal. The range of what is considered as high energy is quite wide, from as low as 50 GeV, to as high as 250 GeV. Our signal reconstruction algorithm always tags higher energy signal events with better efficiency than lower energy events. Thus, all studies done in the following sections were performed using the lowest energy events (50 GeV) in order to provide the efficiency of the BeamCal in the most conservative case.  

            Given that all of the following BeamCal efficiency results are based on the signal tagging algorithm, it is appropriate here to provide an overview of how the reconstruction algorithm works (a more thorough discussion can be found in \cite{bogert_thesis}). The algorithm first populates the pixels of the BeamCalorimeter with the information provided by both the signal electron and the pair background lcdd files, one event at a time. It then performs a sweep across the tenth layer of the BeamCal, aggregating the pixels into clusters called pallets. It selects the pallets with the top 50 highest background-subtracted energies deposited in them. The selected pallets are then extended to include all pixels in the same x/y locations in layers 10 through 40, summing the raw energy of all included pixels to produce a new structure referred to as a cylinder. The 50 cylinders are iterated through. If the background-subtracted energy of any of the cylinders exceeds a particular number of standard deviations of the background energy (sigma cut), then the event is deemed to contain a signal event. The sigma cut is chosen such that no more than ten percent of all events that do \textit{not} contain a signal event falsely identify a signal detection. This algorithm's efficiency in correctly identifying signal electrons, particularly as a function of radius, was then studied and compared across the range of architectural configurations, using three different measures of efficiency.

            By measures of efficiency, I refer to both the instrumental efficiency and geometric efficiency, as well as their product, the total efficiency. Instrumental efficiency is the frequency with which the tagging algorithm correctly identifies events where the high energy electron/positron \textit{hits the BeamCal}. Conversely, geometric efficiency ignores the tagging algorithm entirely, determining only whether or not an electron/positron hit the BeamCal. The product of these two, the frequency of how often the algorithm correctly tags \textit{any} electron/positron (including those which miss the BeamCal), is referred to as the total efficiency. In the following analyses, the total efficiency of the architectures is studied, and then dissected by separately comparing the instrumental and geometric efficiencies.


        \section{Vertex Detector} \label{sect__analysis_vertex}
            For the Vertex Detector, a program was used to pixellate the hits on the detector and then perform occupancy analysis on the resulting pixels as a function of a few different parameters. To reiterate, occupancy here refers to the fraction of pixels hit in some specific area, over some specified integration time. In this study, the Vertex Detector is split into two main components; the Barrel and the Endcap. In figure \ref{fig__vertex}, the Barrel is visible as the cylinders, and the Endcap is visible as the flat circular disks. The occupancy of the Barrel was studied as a function of the radial angle around the z-axis, so that the specified area of the occupancy was a series of angular bins of constant area. To complement this, the occupancy of the Endcap was studied as a function of radius, so that the specified area of the occupancy is a series of concentric radial bins. Initially, both of these were also studied as a function of layer, however as section \ref{sect__analysis_lstar} will discuss, it was only really useful to discuss the first layer of the Barrel and Endcap.

            As for the specified integration time, the vertex detector was initially to be studied with two varied parameters: the accumulation time of the readout electronics, and the size of the pixels. The integration time was set so the Vertex Detector either flushed its buffer every bunch crossing, or would wait the time span of five bunch crossings before flushing its buffer. The pixel size choices were 15x15 microns on the higher resolution end, and 30x30 microns on the lower resolution end. This variety of parameters was ultimately simplified though. It turned out that unless the Vertex Detector used the most conservative of the options (integration over 5 bunch crossings and 30x30 micron pixels), the occupancy levels were almost completely negligible, and there was no variation between the variety of geometric configurations. Thus, all analyses shown in this paper on the Vertex Detector were performed using these most conservative of parameters.


        \section{Results}
            The following three sections look at each of the three primary SiD alterations one at a time, separately analyzing the effects of the alterations on the BeamCal and Vertex Detector. As stated above, the BeamCal is analyzed by looking at the total efficiency, and then breaking the total efficiency down into its component parts, the geometric efficiency and instrumental efficiency. Changes in the geometric efficiency typically indicate that the SiD alteration in question has altered how often signal events actually encounter the BeamCal. Meanwhile, changes in the instrumental efficiency will generally correspond to a change in the level of background energy reaching the BeamCal, which influences the ease with which the reconstruction algorithm can identify a signal event. The Vertex Detector's performance is measured by its occupancy levels in the Endcaps and Barrel, with higher occupancy indicating lower performance. Ultimately, ideal results will show the efficiency in the BeamCal increase towards unity, and the occupancy in the Vertex Barrel and Endcaps drop towards zero.

            \subsection{L star} \label{sect__analysis_lstar}
                The L* is the distance between the SiD interaction point and the final focussing magnet (qd0). The L* was changed from 3.5 meters to 4.1 meters in the latest design report, and the BeamCal's distance from the interaction point was changed with it. Here, the Sidloi3-IR\_Realign geometry corresponds to the new 4.1 meter L*, and the corresponding new BeamCal position at 3.265 meters from the IP. The Sidloi3-IR\_Realign\_preqd0shift geometry corresponds to the old 3.5 meter L* and BeamCal position of 2.775 meters from the IP.

                %beamcal
                \begin{figure}[H] 
                    \includegraphics[height=.4\textheight]{RadialEfficiencyFP_total}
                    \centering
                    \caption{L* = 4.1 m (Sidloi3-IR\_Realign) vs L* = 3.5 m (Sidloi3-IR\_Realign\_preqd0shift). The 4.1 meter L* wins out very slightly.}
                    \label{fig__lstar_beamcal_total}
                \end{figure}
                \begin{figure}[H]
                    \includegraphics[height=.4\textheight]{RadialEfficiencyFP_instrumental}
                    \centering
                    \caption{L* = 4.1 m (Sidloi3-IR\_Realign) vs L* = 3.5 m (Sidloi3-IR\_Realign\_preqd0shift). Note that the two designs are almost perfectly matched. }
                    \label{fig__lstar_beamcal_inst}
                \end{figure}
                \begin{figure}[H]
                    \includegraphics[height=.4\textheight]{RadialEfficiencyFP_geometric}
                    \centering
                    \caption{L* = 4.1 m (Sidloi3-IR\_Realign) vs L* = 3.5 m (Sidloi3-IR\_Realign\_preqd0shift). At low radii, the 4.1 meter L* outperforms the 3.5 meter L* by a wide margin. }
                    \label{fig__lstar_beamcal_geom}
                \end{figure}

                As visible in figure \ref{fig__lstar_beamcal_total}, moving the BeamCal further away from the IP only has a marginal effect on the BeamCal efficiency, except at the innermost radii. Looking at figures \ref{fig__lstar_beamcal_inst} and \ref{fig__lstar_beamcal_geom}, which factorize figure \ref{fig__lstar_beamcal_total}, it is apparent that this is a geometric effect. At a further distance from the IP, the inner radii of the BeamCal corresponds to a lower angle in theta. This means that a number of very low angle electrons are hitting the BeamCal at the 4.1 meter L* position, which missed the BeamCal at the 3.5 meter L*, resulting in more electrons being detected.
                    
                %vertex
                \begin{figure}[H]
                    \includegraphics[height=.4\textheight]{Voccupancy_sidloi3_IR_realign_preqd0shift_5B_ps30_1510212229_brl}
                    \centering
                    \caption{Vertex Barrel 3.5 meter L*. Note that in this and the following plot, layer 0 has much higher occupancy levels than the other layers.}
                    \label{fig__lstar_vertex_brl_3.5}
                \end{figure}
                \begin{figure}[H]
                    \includegraphics[height=.4\textheight]{Voccupancy_sidloi3_IR_realign_5B_ps30_1510211229_brl}
                    \centering
                    \caption{Vertex Barrel 4.1 meter L*. The occupancy levels are almost indistinguishable from that of the previous plot, indicating that the change in L* had almost no effect on the Vertex Barrel.}
                    \label{fig__lstar_vertex_brl_4.1}
                \end{figure}
                \begin{figure}[H]
                    \centering
                    \includegraphics[height=.4\textheight]{Voccupancy_sidloi3_IR_realign_preqd0shift_5B_ps30_1510212229_ecp}
                    \caption{Vertex Endcap 3.5 meter L*. Note that in this and the following plot, the occupancy of layer 0 is nearly identical to that of the other layers.}
                    \label{fig__lstar_vertex_ecp_3.5}
                \end{figure}
                \begin{figure}[H]
                    \includegraphics[height=.4\textheight]{Voccupancy_sidloi3_IR_realign_5B_ps30_1510211229_ecp}
                    \centering
                    \caption{Vertex Endcap 4.1 meter L*. The occupancy here is marginally higher than that of the previous plot (the average is at $0.15*10^{-3}$ pixels hit, compared to the $0.125*10^{-3}$ pixels hit for the 3.5 meter L*)}
                    \label{fig__lstar_vertex_ecp_4.1}
                \end{figure}

                \begin{figure}[H] 
                    \includegraphics[height=.4\textheight]{VradOccupancy_Lstar_brl}
                    \centering
                    \caption{L* = 4.1 m (Sidloi3-IR\_Realign) vs L* = 3.5 m (Sidloi3-IR\_Realign\_preqd0shift) Vertex Barrel L* Occupancy. The occupancy is symmetric in the phi angle, and the two L* configurations show roughly even occupancy levels at all angles.}
                    \label{fig__lstar_vertex_brl}
                \end{figure}
                \begin{figure}[H] 
                    \includegraphics[height=.4\textheight]{VradOccupancy_Lstar_ecp}
                    \centering
                    \caption{L* = 4.1 m (Sidloi3-IR\_Realign) vs L* = 3.5 m (Sidloi3-IR\_Realign\_preqd0shift) Vertex Endcap L* Occupancy. The fraction of pixels that are hit increases at inner radii.  The 4.1 meter L* shows consistently higher occupancy levels than the 3.5 meter L*, but only by small amounts. }
                    \label{fig__lstar_vertex_ecp}
                \end{figure}
                Turning now to the Vertex Detector's relationship with the L*, it can be seen that the effect of the BeamCal movement on the Vertex Detector Occupancy is virtually non-existent. Figures \ref{fig__lstar_vertex_brl_3.5} and \ref{fig__lstar_vertex_brl_4.1} show the total occupancy of the Vertex Barrel for the 3.5 meter L* and 4.1 meter L*, respectively, across all layers. From these it can be seen that the Vertex Barrel occupancy experiences no noticeable change from the L* adjustment. Figures \ref{fig__lstar_vertex_ecp_3.5} and \ref{fig__lstar_vertex_ecp_4.1} show the total occupancy of the Vertex Endcaps, again for the two L* configurations. These two figures show that there is a very slight change to the Endcap, notably that the 4.1 meter L* has slightly higher occupancy than the 3.5 meter L*. Figures \ref{fig__lstar_vertex_brl} and \ref{fig__lstar_vertex_ecp} show this same information in a more detailed fashion. Figure \ref{fig__lstar_vertex_brl} illustrates the near identical Barrel occupancies shown in \ref{fig__lstar_vertex_brl_3.5} and \ref{fig__lstar_vertex_brl_4.1}, noting that neither configurations' occupancy is consistently higher than the other. The reason that the Endcap shows a slight decrease in performance with the 4.1 meter L* is likely the same as that which increased tagging efficiency in the BeamCal. That is, the BeamCal has slightly better angular coverage for low theta particles. This would result in more background particles hitting the BeamCal at low radii, causing slightly higher albedo levels.

                I would also like to take this section to note that, especially in the Barrel, the occupancy levels of the various layers are always worst (or at least no better) for the first layer (layer 0). For this reason, all remaining Vertex plots will focus exclusively on the first layer.

                %conclusion
                Overall, the increase in the length of the L* seems to have been mostly beneficial. Had the Vertex Detector Occupancy or BeamCal Efficiency suffered dramatically, the L* change may have required rethinking. Luckily though, the BeamCal now operates more efficiently at low radii, and the Vertex Detector, while not performing any better, does not perform significantly worse either.


            \subsection{Plug Region}
                The plug region refers to the area of the BeamCal around the beampipe holes. There are three proposed designs for this which remove progressively more of the plug region (refer to figure \ref{fig__beamcal_face}). As the plug region is the location with the highest concentration of background energy deposition, there is a hope that removing this region will reduce the area available for background particles to hit, thus reducing the albedo effect.

                %beamcal
                \begin{figure}[H] 
                    \includegraphics[height=.4\textheight]{RadialEfficiency_total_geom}
                    \centering
                    \caption{Full plug region (Sidloi3-IR\_Realign) vs wedge cut-out plug region
                            (Sidloi3-IR\_Realign\_wedge-cutout) vs circle cut-out plug region
                            (Sidloi3-IR\_Realign\_circle-cutout). The efficiency decreases with 
                            the more zealous cuts, especially at lower radii.}
                    \label{fig__geom_beamcal_total}
                \end{figure}
                \begin{figure}[H]
                    \includegraphics[height=.4\textheight]{RadialEfficiency_instrumental_geom}
                    \centering
                    \caption{Full plug region (Sidloi3-IR\_Realign) vs wedge cut-out plug region (Sidloi3-IR\_Realign\_wedge-cutout) vs circle cut-out plug region (Sidloi3-IR\_Realign\_circle-cutout). The instrumental efficiency for all three designs are nearly identical, showing that the plug region's removal affects only the geometric efficiency.}
                    \label{fig__geom_beamcal_inst}
                \end{figure}
                \begin{figure}[H]
                    \includegraphics[height=.4\textheight]{RadialEfficiency_geometric_geom}
                    \centering
                    \caption{Full plug region (Sidloi3-IR\_Realign) vs wedge cut-out plug region
                            (Sidloi3-IR\_Realign\_wedge-cutout) vs circle cut-out plug region
                            (Sidloi3-IR\_Realign\_circle-cutout). The efficiency decreases dramatically 
                            with the more zealous cuts, especially at lower radii. Combined with the 
                            previous figure's complete lack of difference between the designs, this figure
                            indicates that the efficiency changes are purely geometric in nature.}
                    \label{fig__geom_beamcal_geom}
                \end{figure}

                Obviously, if there is no detector material for a particle to interact with in some region, then the BeamCal will be completely unable to detect anything moving through that region. As expected then, removing part of the BeamCal causes severe drops in efficiency near the removed region, as visible in figure \ref{fig__geom_beamcal_total}. Additionally, the performance drop increases with the more liberal circular cutout. Since nothing altered the behavior of particles or the BeamCal outside of the inner radii, the performance there is unchanged. The effect on the BeamCal is entirely geometric(\ref{fig__geom_beamcal_geom}), with no significant effects on the instrumental efficiency (\ref{fig__geom_beamcal_inst}).
                
                %vertex
                \begin{figure}[H] 
                    \includegraphics[height=.4\textheight]{VradOccupancy_plug_brl}
                    \centering
                    \caption{Full plug region (Sidloi3-IR\_Realign) vs wedge cut-out plug region
                            (Sidloi3-IR\_Realign\_wedge-cutout) vs circle cut-out plug region
                            (Sidloi3-IR\_Realign\_circle-cutout)  Vertex Barrel Occupancy. 
                            The designs with the plug region removed show consistently lower
                            occupancy than the full plug region.}
                    \label{fig__plug_vertex_brl}
                \end{figure}
                \begin{figure}[H] 
                    \includegraphics[height=.4\textheight]{VradOccupancy_plug_ecp}
                    \centering
                    \caption{Full plug region (Sidloi3-IR\_Realign) vs wedge cut-out plug region
                            (Sidloi3-IR\_Realign\_wedge-cutout) vs circle cut-out plug region
                            (Sidloi3-IR\_Realign\_circle-cutout)  Vertex Barrel Occupancy. 
                            The designs with the plug region removed show consistently lower
                            occupancy than the full plug region, with the circle cut-out (which 
                            removes the most area of the BeamCal) having the lowest occupancy.}
                    \label{fig__plug_vertex_ecp}
                \end{figure}
                In contrast to the performance drops in the BeamCal, the Vertex Detector actually sees performance gains. The backscatter effect from the BeamCal into the Vertex Detector is dependent on there being something for the particles to be emitted from. It makes perfect sense then, that removing material from the BeamCal would decrease the occupancy in the Vertex Detector, visible in figures \ref{fig__plug_vertex_brl} and \ref{fig__plug_vertex_ecp}, with the wedge cutout resulting in significant improvements to occupancy levels, and the larger circle cutout resulting in even lower occupancy levels.
                

                %conclusion
                It would certainly prove beneficial to the Vertex Detector's performance if part of the plug region could be removed, but whether or not the losses to the BeamCal are tolerable is still an open question. The events used to analyze them in this study are, as mentioned previously, not real physical events. The particle interaction that this algorithm really needs to be used for is the gamma-gamma to hadron interaction. At the time of writing this, studies on the gamma-gamma to hadron events have begun, but are many months from completion. A more thorough decision on the plug region will be made once the plug region's effect on gamma-gamma to hadron tagging is studied.

            
            \subsection{Anti-DiD}
                The anti-DiD is a magnetic field designed to be overlaid on top of the solenoidal magnetic field, and is intended to redirect background particles down the outgoing beampipe hole. In doing so, we expect to both reduce the reduce background levels in the BeamCal, and also reduce the albedo effect.

                %beamcal
                \begin{figure}[H] 
                    \includegraphics[height=.4\textheight]{RadialEfficiency_total_did}
                    \centering
                    \caption{No anti-DiD field (Sidloi3-IR\_Realign) vs with anti-DiD field 
                            (Sidloi3-IR\_Realign\_anti-did). Both designs use the 4.1 meter L*
                            and the full plug region. Here, the design which includes the anti-DiD
                            field slightly outperforms the design without.}
                    \label{fig__did_beamcal_total}
                \end{figure}
                \begin{figure}[H]
                    \includegraphics[height=.4\textheight]{RadialEfficiency_instrumental_did}
                    \centering
                    \caption{No anti-DiD field (Sidloi3-IR\_Realign) vs with anti-DiD field 
                            (Sidloi3-IR\_Realign\_anti-did). Both designs use the 4.1 meter L*
                            and the full plug region. It is apparent from this plot that the
                            improvements from the anti-DiD field are entirely instrumental.}
                    \label{fig__did_beamcal_inst}
                \end{figure}
                \begin{figure}[H]
                    \includegraphics[height=.4\textheight]{RadialEfficiency_geometric_did}
                    \centering
                    \caption{No anti-DiD field (Sidloi3-IR\_Realign) vs with anti-DiD field 
                            (Sidloi3-IR\_Realign\_anti-did). Both designs use the 4.1 meter L*
                            and the full plug region. The two designs display identical efficiencies,
                            showing that the effect of the anti-DiD field is entirely instrumental.}
                    \label{fig__did_beamcal_geom}
                \end{figure}

                As visible in figure \ref{fig__did_beamcal_total}, the anti-DiD field somewhat improves the BeamCal's instrumental efficiency by  at all radii. The reason for this is likely that the anti-DiD field is funneling a substantial portion of the background energy into the outgoing beam pipe. Without this background energy hitting the BeamCal, the tagging algorithm is able to more easily identify signal events.

                %vertex
                \begin{figure}[H] 
                    \includegraphics[height=.4\textheight]{VradOccupancy_base_brl}
                    \centering
                    \caption{No anti-DiD field (Sidloi3-IR\_Realign) vs with anti-DiD field 
                            (Sidloi3-IR\_Realign\_anti-did) Vertex Barrel Occupancy. Both designs use the 4.1 meter L*
                            and the full plug region. The design which uses the anti-DiD field shows moderately lower occupancy 
                            than the design without.}
                    \label{fig__did_vertex_brl}
                \end{figure}
                \begin{figure}[H] 
                    \includegraphics[height=.4\textheight]{VradOccupancy_base_ecp}
                    \centering
                    \caption{No anti-DiD field (Sidloi3-IR\_Realign) vs with anti-DiD field 
                            (Sidloi3-IR\_Realign\_anti-did) Vertex Barrel Occupancy. Both designs use the 4.1 meter L*
                            and the full plug region. The design which uses the anti-DiD field shows consistently lower occupancy 
                            than the design without at all radii.}
                    \label{fig__did_vertex_ecp}
                \end{figure}

                Similarly to the BeamCal, the Vertex Detector also benefits from the presence of the anti-DiD field. Figures \ref{fig__did_vertex_brl} and \ref{fig__did_vertex_ecp} show that the anti-DiD field reduces the occupancy levels in both the Barrel and Endcap of the Vertex Detector. Just as with the BeamCal tagging efficiency, this improvement results from background particles being sent down the outgoing beam pipe. Because a large number of particles now miss the BeamCal, the albedo is reduced, improving occupancy.

                %Conclusion
                The anti-DiD field certainly accomplishes its intended goal, by improving performance in both the BeamCal and Vertex Detector. However, the BeamCal and Vertex Detector are not the only considerations for the anti-DiD field. Most notably, the anti-DiD is very expensive, costing on the order of millions of dollars (US). As such, the marginal improvements to the BeamCal and Vertex detector are not worth the hefty cost of installing the anti-DiD field. The conclusion of this study and the SiD collaboration as a whole then is that the anti-Did field will \textbf{not} be included in the final design of the SiD.


        \newpage
        \section{Conclusion}
            %L*
            The L* change, with its corresponding change to the BeamCal's distance from the IP, had a very limited, but positive, effect on the performance of both the BeamCal and Vertex Detector. The L*'s effect on the BeamCal was an entirely geometric effect, with less signal events interacting with the BeamCal in the lengthened (4.1 m) L* configuration. The Vertex Detector experienced virtually no change in the Barrel, and only a minor increase in occupancy in the Endcaps with the 4.1 meter L*. The change from an L* of 3.5 meters to an L* of 4.1 meters was therefore mostly beneficial.
            
            %Plug 
            The BeamCal plug region modifications produced inverse results in the BeamCal and Vertex Detector; the performance of the BeamCal suffered substantially as more of the device was cut away, while the Vertex Detector's performance improved across both the Barrel and Endcaps. Due to the ad-hoc nature of the studies performed on the BeamCal, the results of this test will require further study in the future.
            
            %anti-DiD
            The final alteration, the anti-DiD field, marginally improved the performance in both detectors. However, the expected cost of the anti-DiD field means that the small gains in the BeamCal and Vertex Detector are not worth the investment needed for the anti-DiD field.





    \chapter{Appendices}
        \section{GeomConverter Source Code Changelog} \label{sect__geom_changes}
            \begin{verbatim}
all files changed are located in:
lcsim/detector-framework/src/main/java/org/lcsim/geometry/compact/converter/lcdd/


ForwardDetector.java:
    This is the file that determines the BeamCal's geometry, as well as the
    geometry of the polyethylene mask in front of the BeamCal.
    
    It was modified in order to allow the BeamCal and its mask to be alligned
    to the outgoing beam pipe.
    
    The rotation itself was straightforward, requiring only a few lines of
    code towards the end to A) rotate the BeamCal by half the crossing angle
    (0.014 radians, so the BeamCal was rotated by 0.007 radians), and B) 
    shift the BeamCal in x by [zposition*tan(crossing_angle/2)]. The
    shift in x was neccessary because the rotation is about the BeamCal's
    local center, not the global origin. 

    The catch was the two holes in the BeamCal which needed to be subtracted
    from the volume for the beam pipes. This subtraction is done not just in
    every layer, but in every slice of material in every layer. The result is
    effectively the same body of code in three locations to perform the
    rotated subtraction properly (one for the main beamcal, one for each
    layer, and one for each slice). 


util/PhysVol.java:
    This is the class which is actually interpreted by the program as
    belonging to the geometry. When a geometric object in the code is generated, the
    last step is to create a PhysVol object with the geometric object, which
    is then added to the "mother volume" PhysVol object.

    This was modified merely to add two convenience functions. There was
    already a function (setZ) built into it that allowed one to easily set the
    z-position of a PhysVol object. I simply added two more that did the same
    thing for x and y. The function for x (setX) was used for the shift
    mentioned in the ForwardDetector.java, and the setY function was just for
    completeness.


PolyconeSupport.java:
    This is the class which converts, among many other things, the large
    support tube which encompasses the BeamCal and mask.

    It was modified to allow it to be alligned to the outgoing beam pipe.
    Oddly, while the BeamCal requires both a rotation and x position be set,
    the PolyconeSupport only requires rotation, as its axis of rotation
    appears to be the origin.

    While the functionality for alignment is still present in the code, later discussion
    revealed that the support tube should remain alligned to the z-axis, so
    the functionality is not currently being used.


CylindricalEndcapCalorimeter.java:
    This is the class which converts the LumiCal.

    It was modified to allow alignment to the outgoing beam pipe. Like the
    BeamCal, it requires a rotation and shift-in-x parameter to be aligned
    correctly. However, the hole in the center of the LumiCal is generated
    through a different method than the BeamCal holes, which means it does not
    require any special attention.
\end{verbatim}


\section{compact.xml Changelog} \label{sect__compact_changes}
\begin{verbatim}
All numbers in units of mm


Changes made in order to comply with Tom Markiewicz's engineering excel file:

    EcalBarrel inner radius: 1265 -> 1270
    EcalEndcap inner radius: 200 -> 216
    EcalEndcap outer radius: 1294.1 -> 1250

    HcalBarrel inner radius: 1419 -> 1410
    HcalBarrel z from origin to edge: 3018 -> 2940

    HcalEndcap inner radius: 200 -> 216
    HcalEndcap outer radius: 1458.7 -> 1402
    HcalEndcap z-min: 1805 -> 1820

    MuonEndcap zmin: 3028 -> 3030
    MuonEndcap rmin: 200 -> 211

    LumiCal zmin: 1680 -> 1557

    BeamCal outer_r: 129.6 -> 159
    BeamCal inner_z: 2950 -> 3265
    BeamCal slice material Tungsten density 24 thickness: 2.71 -> 2.5
    
    BeamCal support tube rmin: 155 -> 194
    BeamCal support tube rmax: 195 -> 211
    BeamCal support tube zmin: 1820 -> 1700
    BeamCal support tube zmax: 3235 -> 6773

    Forward & Backward M1 support:
        initial: rmin 80 -> 70, rmax 155 -> 100, z 1820 -> 3135
        final:   rmin 137.8 -> 169, rmax 155 -> 194, z 3135 -> 3125

    ForwardLowZ:
        outer_r: 123.9 -> 159
        inner_z: 2820 -> 3135



changes made to allign the BeamCal, mask, and LumiCal to the outgoing beam pipe

    Added "rotation" parameter to LumiCal to allow arbitrary rotation

    Added "allign_to_pipeout" boolean parameter to BeamCal. 
        Does NOT allow arbitrary rotation. The BeamCal is either concentric 
        about the z-axis, or the outgoing beam pipe. This is because, no matter
        how the BeamCal is rotated, the two holes going through it MUST align 
        to the beam pipes. Allowing arbitrary rotation, while possible, would
        have been much more complicated to implement, as it would require both
        the arbitrary angle and beam crossing to be taken into account for 
        these holes.

    Added "allign_to_pipeout" parameter to 'ForwardLowZ' (Polyethylene mask).
        Functions same as the above 

    All BeamPipe inserts (what goes in the BeamCal holes):
        zhalf (half its length): 92.7 -> 87.5
        z: 3042.7 -> 3352.5
        x(+&-): 21.3 -> 23.47



changes made to avoid collisions with other parts of the detector

    SiTrackerEndcap, layer 4, innermost ring radius: 256.716 -> 258.716
        changed to avoid collision with the LumiCal

    BeamPipeLiner: modified to stop before the lumical instead of the tracking
        region

    Forward and Backward Vacuum: adjusted z position to avoid LumiCal

    VXDcableZbackwardOuter: adjusted z position to avoid LumiCal
    VXDcableZbackwardInner: adjusted z position to avoid LumiCal
    VXDcableZforwardOuter: adjusted z position to avoid LumiCal
    VXDcableZforwardInner: adjusted z position to avoid LumiCal
            \end{verbatim}





    \bibliography{ref.bib}    
    \bibliographystyle{apalike}


\end{document}
